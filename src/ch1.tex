\chapter{操作系统的组织结构}

操作系统的一个核心需求是同时支持多个活动。
例如,使用第0章中介绍的系统调用,一个进程可以通过\texttt{fork}创建新的进程。
操作系统必须在进程之间\emph{分时共享(time-share)}计算机的资源。
例如,即使有比硬件处理器数量更多的进程,操作系统也要保证所有的进程都可以继续推进。
操作系统还必须实现进程之间的\emph{隔离(isolation)}。
即,如果一个进程有bug并且失败了,它不应该影响到那些不依赖这个进程的进程。
然而,完全的隔离太过苛刻,因为进程之间可能需要交互,管道就是一个例子。
因此一个操作系统必须满足这三个要求:多路复用,隔离性,交互性。

这一章提供了操作系统是如何组织以满足三个需求的概述。
事实证明有很多方法可以做到这一点,但本文专注于基于\emph{宏内核(monolithic kernel)}的主流设计,很多Unix操作系统都采用了宏内核的方式。
本章通过追踪当xv6开始运行时第一个进程的创建过程来介绍xv6的设计。
同时,本章还简单展示了xv6提供的所有主要抽象的实现、它们如何交互,以及如何满足多路复用、隔离性和交互性。
xv6尽量避免对第一个进程进行特殊处理,而是重用xv6提供的用于进行标准操作的代码。
随后的章节将会详细介绍每一个抽象。

xv6运行在PC平台上的Intel 80386或更新的(“x86”)处理器上,它的大多数底层功能(例如进程的实现)是x86特定的。
本书假设读者已经有过某些架构的机器级编程的经验,并在遇到x86特定的实现时进行介绍。
附录A简单介绍了PC平台。


